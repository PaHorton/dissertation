Space science missions produce increasingly large and complex datasets, placing significant demands on mission operations teams to analyze, prioritize, and react to scientific data in near real-time.
This dissertation presents the development, implementation, and evaluation of novelty detection systems designed to support mission operations across multiple fields of space science, including planetary exploration and astrophysics.
Additionally, this work spans multiple domains, such as tactical planning, integration and testing, and on-board analysis, demonstrating the versatility of novelty detection techniques in various contexts.
Often in mission operations, typical data is well understood, but unexpected data is difficult to identify and react to in a timely manner.
Novelty detection algorithms identify anomalous or atypical features in data by measuring their deviation from a population, offering the potential to improve the efficiency and effectiveness of mission operations teams by triaging data for prioritization in analysis and decision-making.

This dissertation approaches the implementation of novelty detection systems in three primary domains.
The first domain integrates novelty detection into tactical operations for the Mars Science Laboratory Curiosity rover, operationalizing previously developed algorithms to identify novel features in multispectral Mastcam images.
The second domain develops time-series anomaly detection methods for the Roman Space Telescope's Wide Field Instrument, using both statistical and data-driven approaches to locate cosmic ray events and snowballs in detector readout ramps.
The third domain applies novelty detection directly on-board the Astrophysics Stratospheric Telescope for High Spectral Resolution Observations at Submillimeter-wavelengths (ASTHROS) balloon observatory, enabling real-time identification of problematic spectra during flight operations.
ASTHROS represents a major part of this dissertation, as I oversaw the development of the software for the entire readout system, providing the framework for the on-board novelty detection system.
I also collaborated with the ASTHROS science team to analyze analogous data from the Stratospheric Observatory for Infrared Astronomy (SOFIA) to generate maps of [NII] emission spectra and electron density in Carina Nebula, a task ASTHROS will perform during its flight.
The results of this work culminate in a modular software suite for ASTHROS capable of performing real-time data analysis and anomaly detection, providing a powerful tool for future balloon-borne astrophysics missions.